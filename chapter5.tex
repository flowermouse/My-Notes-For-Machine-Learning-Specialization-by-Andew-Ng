\chapter{Tips for Gradient Descent}
\section{Feature scaling}
\noindent
\includegraphics*[width=\textwidth]{images/2.4 (8)}

\vspace{2em}

\noindent
\includegraphics*[width=\textwidth]{images/2.4 (7)}
\noindent
\includegraphics*[width=\textwidth]{images/2.4 (4)}

\section{Normalization}

\subsection*{Mean normalization}
\begin{thmbox}{mean normalization}{mn}
\textbf{for each feature $x_i$}
\begin{equation}
    x_i = \frac{{x_i} - \mu_i}{\max - \min}
\end{equation}
$\mu_i$ is the average value of $x_i$
\end{thmbox}
\noindent
\includegraphics*[width=\textwidth]{images/2.4 (5)}

\subsection*{Z-score normalization}
\begin{thmbox}{z-score normalization}{zn}
\textbf{for each feature $x_i$}
\begin{equation}
    x_i = \frac{{x_i} - \mu_i}{\sigma_i}
\end{equation}
$\mu_i$ is the average value of $x_i$, $\sigma_i$ is the standard deviation of $x_i$
\begin{align*}
    \mu_i &= \frac{1}{m} \sum_{j=1}^{m} x_i^{(j)} \\
    \sigma_i &= \sqrt{\frac{1}{m} \sum_{j=1}^{m} (x_i^{(j)} - \mu_i)^2}
\end{align*}
\end{thmbox}
\noindent
\includegraphics*[width=\textwidth]{images/2.4 (6)}

\section{Check the convergence}
\paragraph*{the cost versus iterations gragh}
A plot of cost versus iterations is a useful measure of 
progress in gradient descent. Cost should always decrease 
in successful runs. The change in cost is so rapid initially, 
it is useful to plot the initial decent on a different scale than the final descent. 
In the plots below, note the scale of cost on the axes and the iteration step.\\
\includegraphics*[width=\textwidth]{images/2.4 (3)}

\section{Choose learning rate}
\noindent
\includegraphics*[width=\textwidth]{images/2.4 (2)}
\noindent
\includegraphics*[width=\textwidth]{images/2.4}

\section{Polynomial regression}
\begin{dfnbox}{Polynomial regression}{pr}
    \hspace{2em}\dfntxt{Polynomial regression} is a form of regression analysis %
in which the relationship between the independent variable $x$ and the %
dependent variable $y$ is modelled as an $n$th degree polynomial in $x$.
\end{dfnbox}

\begin{exbox}{Polynomial regression}{pr}
    \hspace{2em}consider the following data:
    assume $f_{(w, b)} = w \cdot x^2 + b$\par
    \hspace{2em}By looking at the data, we can see that the data is not linear, so we can use polynomial regression to fit the data.\par
    \hspace{2em}So we choose a new feature $x^2$ and apply linear regression to the new feature.(feature engnieering).
\end{exbox}

\subsection*{Feature engineering}
\begin{enumerate}
    \item first, we need to normalize the features.
    \item Usually, it is hard to know which feature to use, so we can try different features and see which one fits the data best.
    \item for example, we can try $x^2$, $x^3$, $x^4$, $x^5$ and so on.
    \item so the hypothesis function will be $f_{(w, b)} = w_1 \cdot x + w_2 \cdot x^2 + w_3 \cdot x^2 + b$
    \item apply linear regression to the new feature.
    \item Eventally, we can get the funtions looks like this for this example: $ f_{(w, b)} = 0.08x + 0.54x^2 + 0.03x^3 + 0.0106 $
    \item Let's review this idea:
    \begin{enumerate}
        \item Intially, the features were re-scaled so they are comparable to each other
        \item less weight value implies less important/correct feature, and in extreme, when the weight becomes zero or very close to zero, the associated feature useful in fitting the model to the data.
        \item above, after fitting, the weight associated with the $x^2$ feature is much larger than the weights for $x$ or $x^3$ as it is the most useful in fitting the data. 
    \end{enumerate}
\end{enumerate}

\vspace{2em}
\begin{notebox}
    \hspace{2em}the essence of polynomial regression is to use linear regression to fit the data, but with the help of feature engineering, we can fit the data better.\par
    \hspace{2em}even when we create new features, we still use linear regression to fit the data.
\end{notebox}